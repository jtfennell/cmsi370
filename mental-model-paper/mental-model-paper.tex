\documentclass[12pt]{article}
\usepackage{float}
\usepackage{cite}
\usepackage{url}

\begin{document}

\begin{abstract}
Abstract not written yet.

\end{abstract}

\title{Wearables and Mental Models}
\author{Jeff Fennell}
\date{October 2014}
\maketitle

\tableofcontents

\section{Introduction}
Wearable devices, also known as wearables or wearable technology, are any form of technology that has been "incorporated into items of clothing and accessories which can comfortably be worn on the body" \cite{wd}. They can take on a variety of forms, including watches, bracelets, glasses, shirts, hats, and more. Medical devices that are meant to be a permanent fixture on the body are also considered wearable devices.

During the past few years, the demand for wearable devices has gone up considerably, and continues to do so. In 2014, the projected value of the market for wearables in the U.S. was nearly 5.2 billion dollars. That market value is expected to more than double by the year 2018, to around 12.7 billion dollars. \cite{wmv}. Due to the increasingly salient presence of wearable devices in our society, improving the methods we use to interact with these devices will become considerably more important. Interaction with wearables is distinctly different than that of typical electronic devices.  Because of the smaller composition of the devices, their nature as extensions of primarily mobile human bodies, and a variety of other characteristics, they present unique challenges to typical paradigms of user-interaction that are worth exploring.

Though there are a multitide of classes of wearable devices currently in existance, for the the purposes of the rest of this paper, in the interest of studying interaction with the technology, wearable devices will primarily refer to wearable computing devices.

\section{Background / Prior literature}

The wearable computing devices that this paper is interested in include wearable watches and glasses, and other accessories that have been converted into wearable computing devices. The following are studies and reports that centered around the common themes of design challenges with wearable devices.

Witt et. al (2006) investigated the challenges of developing a hands-free interface for a computing system \cite{witt}. The authors developed a set of design challenges for hands-free interfaces, and with those constraints in mind, designed a wearable electronic glove that allows a user to control aircraft maintenance software. The paper identifies that small hardware, ability to perform hands-free operation, and the capability to use such an interface with minimal focus, as some of the constraints to designing a hands-free wearable device.

Zhang, et al explored different alternative paradigms of interaction with the Google Glass wearable devices. The authors explore some of the constraints of the mobile human body in designing a user-interface for a wearable device. Additionally, they explored ways in which existing technology in Google glass wearbles, can be improved to further support seamless user-interaction. 

Todi, et al (2014) investigated incorporating wearables into jackets \cite{todi}. The authors attempted to create a user-interface that was seamlessly integrated with already-present clothing, eliminating or reducing the need for the user to focus their attention on a cell phone to interact with it. Ultimately the objective was to develop a wearble interface that would permit secrecy and security in the use of electronics, and reduce the saliency of such electronics in social settings.  The means to acheive this included embedding the wearable devices within clothing and employing the use of distinct, tangible buttons with unique functionality.


\section{Methods}

\section{Discussion}

\section{Conclusion}


\bibliographystyle{plain}
\bibliography{mental-model}

\end{document}