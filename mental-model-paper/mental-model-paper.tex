\documentclass[12pt]{article}
\usepackage{float}
\usepackage{cite}
\usepackage{url}

\begin{document}

\begin{abstract}
Wearable computing devices make up a milti-billion dollar industry that is continuing to grow. Devices are becomming more present within society as more manufacturers move into the wearable industry. With the increasing presence of these devices, it is important to develop accompanying interfaces that utilize accepted user interface principles, to allow as many people as possible to benefit from the technology. By analyzing the design guidelines of two types of wearables, Google Glass and Android Wear, this article attempts to compare usability of wearables to that of conventional computing devices. Accepted principles for good interface design are held up against the physical constraints and intended use of the devices. The goal was to determine which traditional interface principles which are particularly applicable in the context of wearable devices, and which principles applied to computing devices are not as helpful in desigining an interface for a wearable. User interface principles such as Anticipation, Fitts' Law, and Visible Navigation were determined to be applicable in constructing a user-interface for a wearable device. Interaction Design principles such as notifications of system state, and the hotspot concept of fitts' law lose importance in designing interfaces for wearables.

\end{abstract}

\title{Wearables \\ \vspace{2 mm} {\large Usability and Mental Models}}
\author{Jeff Fennell}
\date{October 2014}
\maketitle

\tableofcontents

\section{Introduction}
Wearable devices, also known as wearables or wearable technology, are any form of technology that has been ``incorporated into items of clothing and accessories which can comfortably be worn on the body'' \cite{wd}. Additionally, wearables are capable of delivering information based on ``[sensing] the user's current context'', taking into account location, time, and other factors\cite{star}. They can take on a variety of forms, including watches, bracelets, glasses, shirts, hats, and more. Medical devices that are meant to be a permanent fixture on the body are also considered wearable devices.

During the past few years, the demand for wearable devices has gone up considerably, and continues to do so. In 2014, the projected value of the market for wearables in the U.S. was nearly 5.2 billion dollars. That market value is expected to more than double by the year 2018, to around 12.7 billion dollars. \cite{wmv}. Due to the increasingly salient presence of wearable devices in our society, improving the methods we use to interact with these devices will become considerably more important. Interaction with wearables is distinctly different than that of typical electronic devices.  Because of the smaller composition of the devices, their nature as extensions of primarily mobile human bodies, and a variety of other characteristics, they present unique challenges to typical paradigms of user-interaction that are worth exploring.

Though there are a multitide of classes of wearable devices currently in existance, for the the purposes of the rest of this paper, in the interest of studying interaction with the technology, wearable devices will refer to the wearable computing devices Google Glass, and Android Wear.

\section{Background / Prior literature}

The following are studies and reports that centered around the common themes of design challenges of other wearable devices.

Witt et. al (2006) investigated the challenges of developing a hands-free interface for a computing system \cite{witt}. The authors developed a set of design challenges for hands-free interfaces, and with those constraints in mind, designed a wearable electronic glove that allows a user to control aircraft maintenance software. The paper identifies that small hardware, ability to perform hands-free operation, and the capability to use such an interface with minimal focus, as some of the constraints to designing a hands-free wearable device.

Zhang, et al (2014) explored alternative paradigms of interaction with the Google Glass wearable devices. The authors explore some of the constraints of the mobile human body in designing a user-interface for a wearable device. Additionally, they explored ways in which existing technology in Google glass wearbles can be improved to further support seamless user-interaction. 

Todi, et al (2014) investigated incorporating wearables into jackets \cite{todi}. The authors attempted to create a user-interface that was seamlessly integrated with clothing, eliminating or reducing the need for the user to focus their attention on a cell phone to interact with it. Ultimately the objective was to develop a wearble interface that would permit secrecy and security in the use of electronics, and reduce the saliency of such electronics in social settings.  The means to acheive this included embedding the wearable devices within jackets and employing the use of distinct, tangible buttons with various functionalities.

\section{Methods / Software Design Guidelines} 

In order to determine the specific usability challenges of wearables, it helps to first examine spftware design guidelines for wearables. The following are some of the guidelines for designing interfaces on Google Glass and Android Wear:

\begin{enumerate}
\item{Minimize errors with large buttons}
\item{Make interactions as quick, brainless, and unobtrusive as possible}
\item{Deliver succint information}
\item{Provide contextual notifications}
\end{enumerate}

\subsection{Minimize errors with large buttons}
Wearable devices are by nature, smaller than tradtional computing devices, in order to inconspicously blend in with the wearer. As Witt et al point out, this leads to hardware designers creating a system that uses the least amount of hardware possible \cite{witt}. Because of smaller hardware specifications, the size of display devices for wearables are naturally limited. This can be observed in the screen sizes of the Google Glass and Samsung Wear. The Google Glass display area is the equivalent of a ``25 inch high definition screen from eight feet away'' \cite{goog2}.

Though the user is limited with smaller display sizes on Android Wear platforms, the design documents still indirectly encourages developers to integrate Fitts' law into their design patterns. The formal design principles encourage developers to ``design for big gestures'' \cite{andr}. Rather than forcing the user to try and use small interface components, the concept is to make user interface components as big as possible so the user does not have to worry about being precise with their gestures. This is especially important because watches can potentially be used when active. These bigger buttons that are encouraged by the design documents naturally are wider buttons, which according to Fitts' law, are easier to interact with, and minimize errors. This allows users to focus more on completing tasks, than focusing on the precision of the gestures needed to complete the tasks.

\subsection{Make interactions as quick, brainless, and unobtrusive as possible}
Wearable devices are designed to be an unobtrusive compliment to the user's life. Both Google and Android Wear design documents use this idea to establish a dichotomy between conventional devices and wearables. While desktop computers and mobile phones are meant to be used for extended periods of time, the Google guidelines for these devices state that wearables are meant to be used for quick tasks \cite{andr} \cite{goog}. Because of this, design documents for both systems encourage developers to design software that allow the user to complete tasks with the least amount of gestures possible. Google Wear design principles state that if any task on a Wear device takes the user longer than 5 seconds to perform, the developer should reconsider the implementation of the task \cite{andr}. 

The Android Wear documentation also states that interfaces on the device should be so quick that they never impair the user's ability to engage in a conversation, in terms of maintaining eye contact and train of thought \cite{andr}. This is one of the design paradigms of the Android Wear platform: designing apps that require ``zero or low interaction'' \cite{andr2}.

Google Glass also aims to minimize its saliency in the life of the user, encouraging development of interfaces that ``supplement the user's life without taking away from it'' \cite{goog}. It promotes an interaction paradigm of ``fire-and-forget'', in which the software allows the user to complete tasks and then immediately get back to their life away from the device \cite{goog}.

\subsection{Deliver succint information}
Because of the limited area of screen sizes, and the desire for users to not become absorbed in their wearable, Google Glass and Android wear documentation both state that the only information displayed should be that which is absolutely necessary for the completion of the desired task, eliminating unnecessary words, pictures, etc. Android Wear describes how using this design paradigm of ``low information density'' allows the user to finish interaction with the device quickly, with the efficiency similar to that of checking the time on a normal watch \cite{andr3}. 

\subsection{Provide Contextual Notifications}
In order to prevent the user from having to interact with the device to retrieve information, both the Google Glass and Android wear design documents encourage developers to provide notifications based on the context of the user's environment. Android Wear guidelines explains that Wear apps possess awareness of, and take into account the time, user's physical activity, user's location, and more, in order to deliver information when the user needs it, without the individual having to seek out the information\cite{andr2}. Google glass also aims to provide such a context-driven notification system. The example Google provides in their guidelines for the Glass is a user's shopping list automatically appearing on the display device when the user arrives at a grocery store \cite{goog}.

Both platforms also discuss the importance of moderation in delivering notifications, and making sure they are delivered at the right time. Because of the proximity to the user's body, vibration or audible notifications are much more salient than those of conventional devices. They both address how satisfaction can be lowered if an application delivers too many notifications, or delivers them at inconvenient times \cite{goog} \cite{andr}.

\section{Discussion-Usability Principles and Mental Models}
Wearables are meant to represent a similar mental model to those of traditional computing devices, however they focus on integrating human mobility with these devices. After Reviewing the design guidelines for the two devices as well as the additinal sources, it has become apparent that wearable computing devices and their software are subject to a mostly-different set of design constraints than traditional computing devices. I  propose, based on the device guidelines, that the designers prioritize efficiency as the most important usability metric in their devices. In the interest of making the devices as efficient as possible, and ultimately, as usable as possible, I propose that the main usability issues of the devices that must be accomodated for revolve around 3 inherent characteristics of the devices.

\begin{enumerate}
\item{Smaller size of display hardware}
\item{Proximity/Mobility of Devices}
\end{enumerate}

\subsection{Smaller display hardware}
The size of the display devices integrated in wearables severely limits their usability. Because wearables' displays are merely a fraction of the size of displays on conventional computing devices, they are not optimized for the same type of full-fledged applications found on conventional computing devices. In order to physically display the same amount of information as in such an application, a wearable would have to utilize multitudes of navigation windows, in the form of complex navigation structures. Usability of such software would be hindered by the user having to build and memorize complex mental maps to navigate through the screens.

Additionally, in traditional computing devices, larger screen sizes allow software designers to transfer complex, life-like representations of mental models to the user through the use of content and detail. Because the screen size of wearables are so small, and because the guidelines emphasize efficiency, the amount of content and detail you can and should fit onto the display device is severaly restricted. The Wear and Glass design guidelines' emphasis on delivering only the necessary content, and advising against using small details, instead presenting the user with the big picture, consequently hinder the transmission of complex mental-models to the user.

Bringing the content to the user and reducing screen counts, two concepts described by Bruce Tognazzini's \textbf{visible navigation} principle, are arguably more important in wearables than in traditional computing devices \cite{tog}. The concept of contextual notifications in both Google Glass and Android Wear directly reduce the amount of navigation and number of screens required to access information; the content being delivered in such notifications automatically removes \textit{all} of the navigation required to deliver content to the user. The implementation of contextual notifications also relies strongly on the concept of Tognazzini's principle of \textbf{anticipation}. In order to deliver the \textit{correct} information, the system must be able to accurately predict what the user needs or wants. Being able to gauge \textit{when} the user wants such information, also relies on accurate anticipation on the part of the system to prevent intrusive notifications which could potentially lower user satisfaction \cite{tog}.

Another of Tognazzini's principles is to not sacrifice functionality of a system in lieu of interface simplicity \cite{tog}. This gains much more importance with wearables, as simplicty is one of the core design themes associated with using smaller display devices.

The hotspot aspect of \textit{Fitt's law} also loses importance with wearables, as there are no pointer input devices. There is therefore no ability to right click, or move a pointer to the outside edges of the screen when the application is taking up the entire screen.

\subsection{Proximity/Mobilty of devices}
Because wearables are more intimately connected with the user than conventional computing devices, the principle of notifying the user about the state of the system loses importance. Both of the platforms state in the design guidelines that constant notifications on these devices are much more noticeable, than those on a computer or a mobile phone. An excess of notifications, they state, can lead to irritated, and therefore, unsatisfied users. Additionally, because many of the actions these devices perform occur without the user requesting information, there is far less user interaction that the system needs to acknowledge. 

The principle of allowing the user to undo errors also gains importance with wearables. The guidelines for the platforms demonstrate that the designers intended the devices to accomodate active lifestyles.  The wearables must be able to account for errors users might make while using the devices in an active setting. For example, users might use either device while running, or riding a bike, and could potentially press the wrong button.

\section{Conclusion}
The designers imagined of Android Wear and Google Glass intended the devices to be used as natural, unobtrusive extensions to humans. Because of the smaller nature of the devices, as well as their intimate integration with the human body, the devices are suited for a different interaction style than traditional computing devices. The interaction design paradigm that the designers aimed for involves efficient, simple interactions, that accomodate active lifestyles, and require the users to memorize significantly simpler mental maps than traditional devices. Mental maps of the software are significantly reduced by nature of the devices' contextual notifications, which deliver relevant information to the user based on their environment. Additionally, the designers are limited in their representation and transmission of mental models because of the smaller displays of the devices. The amount of detail they can deliver to the user is drastically limited. As wearable devices continue to evolve, and the technology becomes more mainstream, it will be interesting to see if designers continue to be limited by the smaller displays, or if they come up with a way of further condensing and simplifying their mental models in order to bring wearables to the level of functionality as conventional computing devices.

\bibliographystyle{plain}
\bibliography{mental-model}

\end{document}