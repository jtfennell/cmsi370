\documentclass[12pt]{article}
\usepackage{float}
\usepackage{cite}
\usepackage{url}

\begin{document}

\begin{abstract}
Abstract not written yet.

\end{abstract}

\title{Wearables and Mental Models}
\author{Jeff Fennell}
\date{October 2014}
\maketitle

\tableofcontents

\section{Introduction}
Wearable devices, also known as wearables or wearable technology, are any form of technology that has been "incorporated into items of clothing and accessories which can comfortably be worn on the body" \cite{wd}. They can take on a variety of forms, including watches, bracelets, glasses, shirts, hats, and more. Medical devices that are meant to be a permanent fixture on the body are also considered wearable devices.

During the past few years, the demand for wearable devices has gone up considerably, and continues to do so. In 2014, the projected value of the market for wearables in the U.S. was nearly 5.2 billion dollars. That market value is expected to more than double by the year 2018, to around 12.7 billion dollars. \cite{wmv}. Due to the increasingly salient presence of wearable devices in our society, improving the methods we use to interact with these devices will become considerably more important. Interaction with wearables is distinctly different than that of typical electronic devices.  Because of the smaller composition of the devices, their nature as extensions of primarily mobile human bodies, and a variety of other characteristics, they present unique challenges to typical paradigms of user-interaction that are worth exploring.

Though there are a multitide of classes of wearable devices currently in existance, for the the purposes of the rest of this paper, in the interest of studying interaction with the technology, wearable devices will primarily refer to wearable computing devices.

\section{Background / Prior literature}
The wearable computing devices that this paper is interested in include wearable watches and glasses, and other accessories that have been converted into wearable computing devices.


Witt et. al (2006) investigated the challenges of creating a wearable electronic glove in order to control aircraft maintenance software \cite{witt}. Zhang, et al explored different alternative paradigms of interaction with the Google Glass wearable devices. Todi, et al (2014) investigated incorporating wearables into clothing in order to decrease the conspicuousness of the devices and providing distint, tangible buttons for several tasks. 



\section{Methods}

\section{Discussion}

\section{Conclusion}


\bibliographystyle{plain}
\bibliography{mental-model}

\end{document}